\documentclass[../../main.tex]{subfiles}

\begin{document}

This document is structured in the following way:

\begin{enumerate}
    \item The first chapter is an introduction and overview of the project, setting the context that led to its development, the goals to be achieved, and a general description of its functionality.

    \item The second chapter is a formal description of the domain model and the project through the extensive use of class diagrams and state machine diagrams. The class diagram provides a high-level description of the domain entities and their relationships, while the state machine diagram focuses on modeling the most important entities through their state transitions. All functional requirements and domain assumptions are also presented here to achieve the previously stated goals.

    \item The third chapter presents the non-functional requirements, which are deepened thanks to the description of possible use cases using natural language and sequence or activity diagrams, and illustrates the design constraints.

    \item The fourth and final chapter presents a formal analysis of the model through the use of the open source Alloy language and tool. Some of the configurations created by the tool are included.
\end{enumerate}


\end{document}
