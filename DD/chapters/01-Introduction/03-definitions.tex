\documentclass[../../main.tex]{subfiles}

\begin{document}

\subsection{Definitions}

\begin{description}

    \item[Backend] The part of a computer system or application that 
    is not directly accessed by the user, typically responsible for 
    storing and manipulating data.

    \item[CI Pipeline] A series of steps that introduces automation to improve the process of 
    application development, particularly at the integration and testing phases.
     
    \item[Elasticity] The degree to which a system is able to adapt to workload changes by provisioning and de-provisioning resources in an autonomic manner.

    \item[Layer] A logical structuring mechanism for the elements that make up a software solution.

 
    \item[Pilot project] An initial small-scale implementation that is used to 
    prove the viability of a project idea.

    \item[Tier] A physical structuring mechanism for a system infrastructure.
    
\end{description}

\subsection{Acronyms}

\begin{description}

    \item[CI] Continuous Integration.

    \item[DD] Design Document.
    
    \item[RASD] Requirements Analysis and Specification Document.
    
    \item[UML] Unified Modeling Language.
    
    \item[DREAM] Data-Driven Predictive Farming in Telangana.
    
    \item[G] Goal.
    
    \item[IT device] Information Technology device.
    
    \item[R] Requirement.
    
    \item[SP] Shared Phenomenon.
    
    \item[WP] World Phenomenon.

\end{description}

\subsection{Abbreviations}

\end{document}
